\documentclass{article}

\usepackage[utf8]{inputenc}
\usepackage[T1]{fontenc}
\usepackage{geometry}
\usepackage{amsmath}
\usepackage{amsfonts}
\geometry{a4paper}

\title{Graded Exercise 9}
\author{Duong Le}
\date{}

\begin{document}
\maketitle

\section*{Problem 1}
We have, applying the Master theorem:
\[
\begin{aligned}
c_{crit} = \log_{\sqrt{2}}{2} = 2
\end{aligned}
\]
Also, we have that $O(n^2) = O(n^{c_{crit}}\log^0{n})$ \\\\
$\Rightarrow$ $T(n) = \Theta(n^{c_{crit}}\log{n}) = O(n^{2}\log{n})$

\pagebreak
\section*{Problem 2}
We have:
\[
P_1 = 1 - \frac{1}{2} \cdot 1^2 = \frac{1}{2}  \leq p_0
\]
Also:
\[
\begin{aligned}
&                  & P_{d-1}^2 &\geq 0 \\
&\Rightarrow& P_{d-1}     &\geq P_{d-1} - \frac{1}{2} P_{d-1}^2 = P_{d}\\
\end{aligned}
\]
From the two observations above, we get that:
\[
P_{d} \leq P_{d-1} \leq 1
\]
Also, since $P_{d-1} \leq 1 \rightarrow \frac{1}{2}P_{d-1}^2 \leq P_{d-1}$. Thus, $P_d = P_{d-1} - \frac{1}{2}P_{d-1}^2 \geq 0$ \\\\
$\Rightarrow$ $P_d \in [0, 1]$
\subsection*{Base case}
We have: 
\[
\frac{1}{0 + 1} = 1 \leq P_0
\]
$\Rightarrow$ The base case is correct. 
\subsection*{Induction hypothesis}
Assume that the hypothesis is correct for $d-1$, we need to prove that the hypothesis is also hold for $d$.
\subsection*{Prove}
We know that:
\[
\begin{aligned}
P_{d-1} &\geq \frac{1}{d} \\
P_d &= P_{d-1} - \frac{1}{2}P_{d-1}^2
\end{aligned}
\]
We also know that the function $f(x) = x - \frac{1}{2}x^2$ is increasing in [0, 1], and we have shown that $P_{d-1} \in [0, 1]$, also it is clear that $\frac{1}{d} \in [0, 1]$. Thus, we have:
\[
P_d \geq \frac{1}{d} - \frac{1}{2} \cdot \frac{1}{d^2}
\]
Now, consider the following:
\[
\begin{aligned}
&                      & 2d^2 &\geq d^2 + d = d(d + 1) \\
&\Leftrightarrow& \frac{1}{2d^2} &\leq \frac{1}{d(d+1)} \\
&\Leftrightarrow& -\frac{1}{2d^2} &\geq -\frac{1}{d(d+1)}
\end{aligned}
\]
Combine the two above results, we get:
\[
P_d \geq \frac{1}{d} - \frac{1}{d(d+1)} = \frac{1}{d+1}
\]
$\Longrightarrow$ Thus, the proof is complete.


\pagebreak
\section*{Problem 3}
\subsection*{b.}
For each call of FASTMINCUT, the algorithm will make two recursive call on the graph of size $n/\sqrt{2}$, along with one call to the CONTRACT algorithm. \\\\
Also, in the first iteration, the runtime of the CONTRACT algorithm is $O(( n - n/\sqrt{2})^2) = O(n^2)$. Thus, we can write the recurrence of $T(n)$ as:
\[
T(n) = 2T \left( \frac{n}{\sqrt{2}} \right) + O(n^2)
\]
And in problem 1, I have shown that the results of $T(n)$ is $O(n^{2}\log{n})$. \\\\
$\Rightarrow$ The runtime of the algorithm is $O(n^{2}\log{n})$



\end{document}