\documentclass{article}

\usepackage[utf8]{inputenc}
\usepackage[T1]{fontenc}
\usepackage{geometry}
\usepackage{amsmath}
\usepackage{amsfonts} 
\usepackage{algpseudocode}
\usepackage{algorithm}
\geometry{a4paper}

\title{Graded Exercise 4}
\author{Duong Le}
\date{}

\begin{document}
\maketitle

\section*{Problem 1}
\subsection*{a.}
Let denote the number of edge in a tree as $m$. \\\\
Consider, in a tree, the distance between two vertices $\{u, v\}$ will be $m$ at most. This distance is achieved in case the tree is in a shape of a $line$, where the distance of two degree 1 nodes is $m$. \\\\
On the other hand, in a graph, those two node can be neighbour, which means $d_{G}(u, v) = 1$. Additionally, since we are considering spanning tree, the number of vertices in the tree is equal to the number of vertices in the graph, let denote this number as $n$. We also know that in a tree, the number of egde is equal to the number of vertices minus 1. \\\\
$\Rightarrow$ The worst case value of $t$ is $n-1$.

\pagebreak
\subsection*{b.}
Consider an abitrary edge $(u, v) \in E$. Since this is an edge, $d_{G}(u, v) = 1$. Additionally, let the length of the shortest cycle that contains the vertices $u$ and $v$ denoted as $k$. Now, if we remove the edge $(u, v)$ to obtain the subgraph, we get that $d_{G'}(u, v) = k - 1$. Since the girth of the graph is strictly larger than $t + 1$, we have the following:
\[ 
\begin{aligned}
&                      &     k &> t + 1 \\
&\Leftrightarrow& k - 1 &> t  \\
&\Rightarrow& d_{G'}(u, v) &> t \cdot d_{G}(u, v) \nonumber 
\end{aligned}
\]
We can see that the condition of $t$-spanner is violated. \\\\
$\Rightarrow$ A graph with the girth strictly larger than $t+1$ has no proper subgraph that is a $t$-spanner.

\pagebreak
\subsection*{c.}
Consider an edge $\{u, v\} \in G$. If $d_{G}(u, v) \neq w(u, v)$, we can always remove the edge $\{u, v\}$ without changing the distance of $u$ and $v$. And every spanner of the resulting graph is also a spanner of the original graph. \\\\
So we get that $d_{G}(u, v) = w(u, v)$. Now, consider the edge $\{u, v\}$, there are two cases regarding the distance of $u$ and $v$ in the subgraph $G'$: 
\begin{align}
        \begin{cases}
            d_{G'}(u, v) \leq t \cdot w(u, v) \\
            d_{G'}(u, v) > t \cdot w(u, v)
        \end{cases}
        \nonumber
\end{align} 
In the former case, the property of $t$-spanner is already satisfied. And in the second case, $\{u, v\}$ is added to $G'$, making $d_{G'}(u, v) = w(u, v)$. Clearly, for every postive interger $t$, $w(u, v) \leq t \cdot w(u, v)$, thus satisfying the spanner requirement. \\\\
$\Rightarrow$ The algorithm yields a $t$-spanner.

\pagebreak
\subsection*{d.}
Assume that $G'$ has girth less than $t+1$. This means that the graph has a cycle which length is at most $t$. Let the last edge added to the said cycle be $\{u, v\}$. Since the edges are ordered, $w(u, v)$ should be the largest among the cycle edges. Also, the cycle up to this point can only have at most $t - 1$ edges, otherwise it will create a cycle of length $t+1$. But in that case, we also have the following:
\[
d_{G'}(u, v) \leq (t-1) \cdot w(u, v) < t \cdot w(u, v)
\]
But by the algorithm, $\{u, v\}$ can only be added if $d_{G'}(u, v) > t \cdot w(u, v)$. Thus, we have a contradiction. \\\\
$\Rightarrow$ The girth of $G'$ is at least $t+1$.



\end{document}